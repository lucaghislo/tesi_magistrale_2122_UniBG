\unnumberedchapter{Abstract} 
\chapter*{Abstract} 
According to what is supported by some astrophysics theories, about 90\% of the mass of the universe is made by a hypothetical form of matter named Dark Matter and its discovery is one of the main scientific objectives of the 21st century in the field of physics research. Dark matter is not directly observable, since, unlike normal matter, it does not emit electromagnetic radiation and only manifests itself trough gravitational effects. 
In this context, the General AntiParticle Spectrometer project stands as a modern approach for the indirect search of dark matter trough the detection of cosmic antideuterium. The experiment relies on two detectors: a time-of-flight system, which tags candidate events for the detector to save and makes a precise velocity measurement, and a tracker system based on lithium-drifted silicon detectors, which serves as the target and tracker for the initial cosmic-ray particle and its annihilation products. An Application Specific Integrated Circuit designed in a commercial \SI{180}{\nano\meter} CMOS technology is currently employed for the readout of the lithium-drifted silicon detectors and will be used for the first flight of the experiment foreseen for late 2023. This thesis work is aimed at the characterisation and experimental evaluation of the flight components that will be used for the assembly of flight modules of the Lithium-drifted silicon, Si(Li), tracker.\\\\

\noindent
\textbf{Keywords: } GAPS, Spectrometer, Dark matter, Antiparticle, Si(Li), ASIC

