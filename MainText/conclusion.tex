\chapter*{Conclusions}

\par
GAPS (General Antiparticle Spectrometer) is an international experiment aimed at detecting antideuterons and other low-energy particles emitted by cosmic rays. The charge released by an incident particle in the GAPS Si(Li) detectors is read out by an Application Specific Integrated Circuit (ASIC) named \textit{SLIDER32}. It is installed on a custom designed Front-End Board (FEB) and it is connected to 4 Si(Li) detectors, housed in a metal frame that comprises the basic module on which the GAPS tracker is based on.

\par
The main arguments discussed in this thesis work are relevant to the characterisation and experimental evaluation of all flight items that constitute a single Si(Li) tracker module, in order to verify their correct functioning before final integration of the assembled modules into the flight version of the experiment, that will be completed by the end of 2022.

\par
First, the SLIDER32 ASIC in the version intended for flight was successfully tested by means of a custom-designed test board in order to verify its operation at varying temperatures, and the results were consistent with what was predicted by the simulations and observed during the testing of the prototype version of the chip called \textit{pSLIDER32}.

\par
The main work was aimed at testing and validating the flight items constituting the tracker, namely FEBs, dummy-1 FEBs, shields, flex-rigid boards and termination connectors. Each of them was tested individually and subjected to validation procedures to verify their correct functioning and adherence to design standards. The validated components were finally shipped to Columbia University in New York and then assembled at the MIT Bates facility in Middleton, Massachusetts.

\par
Finally, an experiment on a complete module equipped with Si(Li) sensors was conducted to validate the operation of the readout electronics by means of a trigger signal provided by an external scintillator for the purpose of detecting cosmic muons. The results of this last experiment are encouraging and have made it possible to confirm the soundness of the design choices made at various levels in the development of all components constituting the readout electronics.

\par
Future work includes the implementation of an accurate calibration technique for all Si(Li) tracker modules when the assembly operations currently underway are completed.  The calibration procedure will be aimed at precisely defining the correct configuration of the fine threshold values for every channel of each of the tracker modules. In this regard, the measurements carried out in Chapter 3 using both the Americium 241 source and the scintillator for detecting cosmic muons have provided useful information for this purpose and will be extended and studied in greater detail in the future.