\chapter*{Introduction}  % Name of the unnumbered section

\par
According to what is supported by some astrophysics theories, about 90\% of the mass of the universe is made by a hypothetical form of matter named Dark Matter and its discovery is one of the main scientific objectives of the 21st century in the field of physics research. Dark Matter is not directly observable, since, unlike normal matter, it does not emit electromagnetic radiation and only manifests itself through gravitational effects. 

\par
In this context, the General AntiParticle Spectrometer (GAPS) project stands as a modern approach to the indirect search of Dark Matter through the detection of cosmic antideuterium. The experiment is developed by an international collaboration that includes Japanese, US and Italian institutes, and it is funded by NASA, INFN, ASI, JAXA and other research and academic institutions. The instrument relies on two detectors: a time-of-flight system, which tags candidate events for the detector to save and makes a precise velocity measurement, and a tracker system based on lithium-drifted silicon, Si(Li), detectors, which serves as the target and tracker for the initial cosmic-ray particle and its annihilation products. An Application Specific Integrated Circuit (ASIC) called \textit{SLIDER32} designed in a commercial \SI{180}{\nano\meter} CMOS technology is currently employed for the readout of the lithium-drifted silicon detectors and will be used for the first flight of the experiment scheduled for late 2023 from the the McMurdo station in Antarctica. The experimental characterisation of the ASIC is the focus of the thesis work discussed  in the following pages.

\par
The manuscript is organised as follows. \hyperref[ch1]{Chapter \ref{ch1}} describes the characterisation work that has been performed on the ASIC using a purposely built test board in order to carry out an analysis of the performance of the integrated circuit under varying temperature conditions. The measurements were performed to highlight the variations to which the transfer function, current reference, global threshold voltage and electronic noise are subjected.

\par
\hyperref[ch2]{Chapter \ref{ch2}} analyses the test and validation work that has been carried out on the flight components of the lithium-drifted silicon tracker of the GAPS experiment. The test procedures for each of the items and the results obtained are reported, detailing for each component the expected performance compared to that obtained during the test procedure.

\par
Lastly, \hyperref[ch3]{Chapter \ref{ch3}} reports the results obtained during the experimental demonstration of cosmic muon detection using a fully assembled Si(Li) tracker module, using the ArduSiPM cosmic ray and nuclear radiation detector as a trigger for the readout electronics. In addition, this Chapter discusses the characterisation of the fully assembled Si(Li) tracker module, which was also carried out using an Americium 241 source. A full description of the setup used during the experiment and the results obtained are given in this Chapter.

\par
A detailed description of the GAPS experiment, its scientific aims in the context of physics research and the instrument on which it is based are provided in \hyperref[appendixGAPSintro]{Appendix \ref{appendixGAPSintro}}, with \hyperref[appendixGAPSdarkmatter]{Section \ref{appendixGAPSdarkmatter}} providing a brief introduction to Dark Matter.